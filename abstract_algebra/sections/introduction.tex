\section{Introduction}

\begin{definition}
  A \textbf{operation}, $\star$, over a set $S$ is a mapping,
  \begin{equation*}
    \star : S \times S \to S
  \end{equation*}
  that assigns to each $(a,b) \in S$ a unique element $c = a \star b \in S$.
\end{definition}
\begin{example}
  Take the set $\mathbb{R}$ where the operations $+$ and $\cdot$  are
  well defined over $\mathbb{R}$, i.e., $+$ is defined as $a + b$ for
  $a,b \in \mathbb{R}$ and $\cdot$ as $a \cdot b$. Therefore, each
  pair $(a,b) \in \mathbb{R}$ is given an element $a + b$ and $a \cdot b$.
\end{example}
\begin{example}
  An example of a set where the operation $+$ fails is the set of all
  matrices with real-valued entries, $M(\mathbb{R})$. This is
  because, matrix addition only works when two matrices have the same
  number number of rows and columns.
\end{example}
\begin{itemize}[left=0pt]
  \item Say that $\star$ is a valid operation on $S$, then $S$ is
    said to be closed under $\star$ but a subset of $S$ may not be,
    e.g., the set of nonzero real numbers $\mathbb{R}^{\star}$. This
    can be seen easily with the fact that $1 \in \mathbb{R}^{\star}$
    and $-1 \in \mathbb{R}^{\star}$ but $1 + (-1) = 0 \notin
    \mathbb{R}^{\star}$.
  \item Formally, we call this an \textbf{induced operation} where
    $\star$ is an operation on $S$ and $H \subseteq S$. Here $H$ is
    closed under $\star$ only if for all $a,b \in H,\ a \star b \in H$.
\end{itemize}
\begin{example}
  Let $H = \{n^{2} \mid n \in \mathbb{Z}^{+}\}$.
  \begin{enumerate}[left=0pt]
    \item Addition: Take $n_{1} = 1 \in \mathbb{Z}^{+}$ and $n_{2} =
      5 \in \mathbb{Z}^{+}$, then it is obvious that $1 \in H$ and
      $25 \in H$ but $1 + 26 \notin H$. Therefore, addition fails on $H$.
    \item Multiplication: Take two integers $p, q \in H$ which are
      defined as $p = n^{2}$ and $q = m^{2}$ where $n, m \in
      \mathbb{Z}^{+}$. The product $p \cdot q = (n^{2}) \cdot (m^{2}) = (n
      \cdot m)^{2} \in H$ since $n \cdot m \in \mathbb{Z}^{2}$.
      Therefore, $H$ is closed under $\cdot$.
  \end{enumerate}
\end{example}
\begin{definition}
  An operation $\star$ on $S$ is \textbf{commutative} if $a \star b =
  b \star a$ for all $a,b \in S$.
\end{definition}
\begin{definition}
  An operation $\star$ on $S$ is \textbf{associative} if $(a \star b) \star c
  = a \star (b \star c)$ for all $a,b,c \in S$.
\end{definition}
\begin{itemize}[left=0pt]
  \item If $\star$ is not associative then expressions like $a \star b
    \star c$ are said to be \textit{ambiguous} as the result of $a
    \star b \star c$ depends on the grouping order, i.e., $(a \star
    b) \star c$ and $a \star (b \star c)$ yield different results
    from each other.
\end{itemize}
\begin{definition}
  For any set $S$ and functions $f, g$ that map $S$ into $S$, the
  composition $f \circ g$ is defined as the function mapping $S$ into
  $S$ such that $(f \circ g)(x) = f(g(x))$ for all $x \in S$.
\end{definition}
\begin{theorem}[Associativity of Composition]
  Let $S$ be a set and let $f,g, \text{ and } h$ be functions mapping
  $S$ into $S$, then $f \circ (g \circ h) = (f \circ g) \circ h$.

  \begin{proof}
    Given a set $S$ where $x \in S$ let $f, g, \text{ and } h$ be
    functions that map $S$ into $S$. Solving the left side of $(f
    \circ g)(x) = f(g(x))$ results in
    \begin{equation*}
      (f \circ (g \circ h))(x) = f \circ ((g \circ h)(x)) = f(g(h(x)).
      \end{equation*}
      Similarly, solving the right side yields
      \begin{equation*}
        ((f \circ g) \circ h)(x) = (f \circ g)(h(x)) = f(g(h(x)).
        \end{equation*}
        Thus, the composition of functions is associative.
      \end{proof}
    \end{theorem}
